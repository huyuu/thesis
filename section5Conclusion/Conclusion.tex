% section 5 conclusion
In this thesis, we have dedicated in developing a shielding system which is able to operate under high magneitc field environment and have enough "cloak" ability.
The target is set to be spaceusage in which the system is required to work under 1 T field and not weakening the outer field due to the invasion of comic rays.
During our investigation,
we have proposed the Electromagnetic Induction Type Magnetic Cloak,
which combines the high temperature superconductor tape and the ferromagnet to achieve shielding and cloaking ability.
Briefly speaking, the structure is to place a ferromagnets on the edge of the a superconducting coil,
of which the detail is shown in chapter 2.

Through our study on the effectiveness of EIMC, denoted in chapter 3,
we have prooved three points:
\begin{enumerate}
  \item The time constant should be at least a few years, which allowing the system to shield high stable magnetic fields.
  \item The scaled down model shows a shielding ability of over 90\%, from which shielding ability near 100\% in full scale model can be expected.
  \item Inserting ferromagnet is able to reinforce the external field near the coil edge, which reduces the probability of comic rays penetrating into the space craft.
\end{enumerate}
Throug the study in chapter 4, we are able to reveal some guidelines on the construction according to the optimization calculation, which are:
\begin{enumerate}
  \item For the superconductor windings, the shielding rates can be achieved by placing more turns on the edge. On the other hand,
  reducing the length of the coil may achieve a higher cloaking property,
  reinforcing the outer field nearby.
  \item For the ferromagnet, it is considered the best to be placed almost flat on the edge of the coil.
  Since the system works under high fields, the ones having maximum magnetization is considered better when chosing between different ferromagnets made of different material,
  while the permeability is the secondary parameter with a lower priority.
\end{enumerate}

Finally, we are able to answer the foundemantal question:
Can magnetic cloak under high fields be achieved?
Our answer is: conditionally afirmative.
Although good shielding ability can be expected, the property of not disturbing external field is limited by the maximum magnetization.
The stronger magnet is applied, the higher cloaking ability can be achieved.
